% Options for packages loaded elsewhere
\PassOptionsToPackage{unicode}{hyperref}
\PassOptionsToPackage{hyphens}{url}
%
\documentclass[
]{article}
\usepackage{lmodern}
\usepackage{amssymb,amsmath}
\usepackage{ifxetex,ifluatex}
\ifnum 0\ifxetex 1\fi\ifluatex 1\fi=0 % if pdftex
  \usepackage[T1]{fontenc}
  \usepackage[utf8]{inputenc}
  \usepackage{textcomp} % provide euro and other symbols
\else % if luatex or xetex
  \usepackage{unicode-math}
  \defaultfontfeatures{Scale=MatchLowercase}
  \defaultfontfeatures[\rmfamily]{Ligatures=TeX,Scale=1}
\fi
% Use upquote if available, for straight quotes in verbatim environments
\IfFileExists{upquote.sty}{\usepackage{upquote}}{}
\IfFileExists{microtype.sty}{% use microtype if available
  \usepackage[]{microtype}
  \UseMicrotypeSet[protrusion]{basicmath} % disable protrusion for tt fonts
}{}
\makeatletter
\@ifundefined{KOMAClassName}{% if non-KOMA class
  \IfFileExists{parskip.sty}{%
    \usepackage{parskip}
  }{% else
    \setlength{\parindent}{0pt}
    \setlength{\parskip}{6pt plus 2pt minus 1pt}}
}{% if KOMA class
  \KOMAoptions{parskip=half}}
\makeatother
\usepackage{xcolor}
\IfFileExists{xurl.sty}{\usepackage{xurl}}{} % add URL line breaks if available
\IfFileExists{bookmark.sty}{\usepackage{bookmark}}{\usepackage{hyperref}}
\hypersetup{
  pdftitle={ezmmek tutorial},
  hidelinks,
  pdfcreator={LaTeX via pandoc}}
\urlstyle{same} % disable monospaced font for URLs
\usepackage[margin=1in]{geometry}
\usepackage{color}
\usepackage{fancyvrb}
\newcommand{\VerbBar}{|}
\newcommand{\VERB}{\Verb[commandchars=\\\{\}]}
\DefineVerbatimEnvironment{Highlighting}{Verbatim}{commandchars=\\\{\}}
% Add ',fontsize=\small' for more characters per line
\usepackage{framed}
\definecolor{shadecolor}{RGB}{248,248,248}
\newenvironment{Shaded}{\begin{snugshade}}{\end{snugshade}}
\newcommand{\AlertTok}[1]{\textcolor[rgb]{0.94,0.16,0.16}{#1}}
\newcommand{\AnnotationTok}[1]{\textcolor[rgb]{0.56,0.35,0.01}{\textbf{\textit{#1}}}}
\newcommand{\AttributeTok}[1]{\textcolor[rgb]{0.77,0.63,0.00}{#1}}
\newcommand{\BaseNTok}[1]{\textcolor[rgb]{0.00,0.00,0.81}{#1}}
\newcommand{\BuiltInTok}[1]{#1}
\newcommand{\CharTok}[1]{\textcolor[rgb]{0.31,0.60,0.02}{#1}}
\newcommand{\CommentTok}[1]{\textcolor[rgb]{0.56,0.35,0.01}{\textit{#1}}}
\newcommand{\CommentVarTok}[1]{\textcolor[rgb]{0.56,0.35,0.01}{\textbf{\textit{#1}}}}
\newcommand{\ConstantTok}[1]{\textcolor[rgb]{0.00,0.00,0.00}{#1}}
\newcommand{\ControlFlowTok}[1]{\textcolor[rgb]{0.13,0.29,0.53}{\textbf{#1}}}
\newcommand{\DataTypeTok}[1]{\textcolor[rgb]{0.13,0.29,0.53}{#1}}
\newcommand{\DecValTok}[1]{\textcolor[rgb]{0.00,0.00,0.81}{#1}}
\newcommand{\DocumentationTok}[1]{\textcolor[rgb]{0.56,0.35,0.01}{\textbf{\textit{#1}}}}
\newcommand{\ErrorTok}[1]{\textcolor[rgb]{0.64,0.00,0.00}{\textbf{#1}}}
\newcommand{\ExtensionTok}[1]{#1}
\newcommand{\FloatTok}[1]{\textcolor[rgb]{0.00,0.00,0.81}{#1}}
\newcommand{\FunctionTok}[1]{\textcolor[rgb]{0.00,0.00,0.00}{#1}}
\newcommand{\ImportTok}[1]{#1}
\newcommand{\InformationTok}[1]{\textcolor[rgb]{0.56,0.35,0.01}{\textbf{\textit{#1}}}}
\newcommand{\KeywordTok}[1]{\textcolor[rgb]{0.13,0.29,0.53}{\textbf{#1}}}
\newcommand{\NormalTok}[1]{#1}
\newcommand{\OperatorTok}[1]{\textcolor[rgb]{0.81,0.36,0.00}{\textbf{#1}}}
\newcommand{\OtherTok}[1]{\textcolor[rgb]{0.56,0.35,0.01}{#1}}
\newcommand{\PreprocessorTok}[1]{\textcolor[rgb]{0.56,0.35,0.01}{\textit{#1}}}
\newcommand{\RegionMarkerTok}[1]{#1}
\newcommand{\SpecialCharTok}[1]{\textcolor[rgb]{0.00,0.00,0.00}{#1}}
\newcommand{\SpecialStringTok}[1]{\textcolor[rgb]{0.31,0.60,0.02}{#1}}
\newcommand{\StringTok}[1]{\textcolor[rgb]{0.31,0.60,0.02}{#1}}
\newcommand{\VariableTok}[1]{\textcolor[rgb]{0.00,0.00,0.00}{#1}}
\newcommand{\VerbatimStringTok}[1]{\textcolor[rgb]{0.31,0.60,0.02}{#1}}
\newcommand{\WarningTok}[1]{\textcolor[rgb]{0.56,0.35,0.01}{\textbf{\textit{#1}}}}
\usepackage{graphicx}
\makeatletter
\def\maxwidth{\ifdim\Gin@nat@width>\linewidth\linewidth\else\Gin@nat@width\fi}
\def\maxheight{\ifdim\Gin@nat@height>\textheight\textheight\else\Gin@nat@height\fi}
\makeatother
% Scale images if necessary, so that they will not overflow the page
% margins by default, and it is still possible to overwrite the defaults
% using explicit options in \includegraphics[width, height, ...]{}
\setkeys{Gin}{width=\maxwidth,height=\maxheight,keepaspectratio}
% Set default figure placement to htbp
\makeatletter
\def\fps@figure{htbp}
\makeatother
\setlength{\emergencystretch}{3em} % prevent overfull lines
\providecommand{\tightlist}{%
  \setlength{\itemsep}{0pt}\setlength{\parskip}{0pt}}
\setcounter{secnumdepth}{-\maxdimen} % remove section numbering

\title{\emph{ezmmek} tutorial}
\author{}
\date{\vspace{-2.5em}}

\begin{document}
\maketitle

\hypertarget{introduction}{%
\subsection{1 Introduction}\label{introduction}}

Extracellular enzyme assays are a widely-used method to probe the
interactions between microbes and complex organic matter. While enzyme
assays are popular techniques, there remains a need to standardize the
exact methodologies used by practitioners. Here we describe
\emph{ezmmek} (Easy Michaelis-Menten Enzyme Kinetics), a package
designed to compare common environmental enzyme assay protocols.
\emph{ezmmek} is capable of calibrating, calculating, and plotting
enzyme activities as they relate to the degradation of synthetic
substrates. At present, \emph{ezmmek} contains functions to compare two
enzyme assay protocols. The first, as outlined by German \emph{et al.}
(2011), accounts for quenching by considering the contribution of each
individual component, such as enzyme substrate, buffer standard and
homogenate. The second protocol, used by Steen and Arnosti (2011),
assumes that the interaction between fluorophore/chromophore standard
and homogenate will approximately equal the interaction between that
standard's corresponding substrate and homogenate, and result in the
same bulk quenching. \emph{ezmmek} contains datasets collected from the
same samples via each protocol.

\hfill\break

\hypertarget{german-protocol-equations}{%
\subsubsection{1.1 ``German'' protocol
equations:}\label{german-protocol-equations}}

\[\scriptsize (1)~Activity~(nmol~g⁻¹~h⁻¹)~=~\frac{Net~Fluorescence~*~Buffer~Volume~(mL)}{Emission~Coefficient~*~ Homogenate~Volume~(mL)~*~Time~(h)~*~Soil~Mass~(g)}\]

\scriptsize *For water samples, `Soil Mass' and `Homogenate Volume'
should both be set to `1'.

\hfill\break
\[\scriptsize (2)~Net~Fluorescence~=~\frac{Assay~-~Homogenate~Control}{Quench~Coefficient}~-~Substrate~Control\]\\
\[\scriptsize (3)~Emission~Coefficient~(fluorescence~nmol⁻¹)~=~\frac{Standard~Curve~Slope~(in~presence~of~homogenate)~[\frac{Fluorescence}{\frac{nmol}{ml}}]}{Assay~Volume~(mL)}\]\\
\[\scriptsize (4)~Quench~Coefficient~=~\frac{Slope~of~Standard~Curve~(in~presence~of~homogenate)}{Slope~of~Standard~Curve~(in~presence~of~buffer)}\]

\hfill\break
\hfill\break

\hypertarget{steen-protocol-equations}{%
\subsubsection{1.2 ``Steen'' protocol
equations:}\label{steen-protocol-equations}}

\[\scriptsize (5)~Activity~(nmol~g⁻¹~h⁻¹)~=~\frac{m[Concentration~of~Fluorophore~(nmol)~\sim~Time~(h)] }{Soil~Mass~(g)}\]

\scriptsize *For water samples, `Soil Mass' should be set to `1'.

\hfill\break
\[\scriptsize (6)~Concentration~of~Fluorophore~(nmol)~=~\frac{Assay~Fluorescence~-~Kill ~Control~-~Intercept~of~Standard~Curve~(in~presence~of~homogenate)}{Slope~of~Standard~Curve~(in~presence~of~homogenate)}\]\\

\hypertarget{installation}{%
\subsection{2 Installation}\label{installation}}

The latest versions of \emph{ezmmek} are available through CRAN and
GitHub.

\hypertarget{cran}{%
\subsubsection{2.1 CRAN}\label{cran}}

\begin{Shaded}
\begin{Highlighting}[]
\KeywordTok{install.packages}\NormalTok{(}\StringTok{"ezmmek"}\NormalTok{)}
\end{Highlighting}
\end{Shaded}

\hypertarget{github}{%
\subsubsection{2.2 GitHub}\label{github}}

\hypertarget{install-and-load-devtools}{%
\paragraph{\texorpdfstring{2.2.1 Install and load
\emph{devtools}}{2.2.1 Install and load devtools}}\label{install-and-load-devtools}}

\begin{Shaded}
\begin{Highlighting}[]
\KeywordTok{install.packages}\NormalTok{(}\StringTok{"devtools"}\NormalTok{)}
\KeywordTok{library}\NormalTok{(devtools)}
\end{Highlighting}
\end{Shaded}

\hypertarget{install-ezmmek}{%
\paragraph{\texorpdfstring{2.2.2 Install
\emph{ezmmek}}{2.2.2 Install ezmmek}}\label{install-ezmmek}}

\begin{Shaded}
\begin{Highlighting}[]
\KeywordTok{install\_github}\NormalTok{(}\StringTok{"ccook/ezmmek"}\NormalTok{)}
\end{Highlighting}
\end{Shaded}

\hypertarget{required-packages}{%
\paragraph{2.2.3 Required packages}\label{required-packages}}

\begin{Shaded}
\begin{Highlighting}[]
\NormalTok{assertable}
\NormalTok{dplyr}
\NormalTok{magrittr}
\NormalTok{nls2}
\NormalTok{purrr}
\NormalTok{rlang}
\NormalTok{tidyr}
\end{Highlighting}
\end{Shaded}

\hypertarget{how-to-use}{%
\subsection{3 How to use}\label{how-to-use}}

\hypertarget{load-ezmmek}{%
\subsubsection{\texorpdfstring{3.1 Load
\emph{ezmmek}}{3.1 Load ezmmek}}\label{load-ezmmek}}

\begin{Shaded}
\begin{Highlighting}[]
\KeywordTok{library}\NormalTok{(ezmmek)}
\end{Highlighting}
\end{Shaded}

\hypertarget{example-datasets}{%
\subsubsection{3.2 Example datasets}\label{example-datasets}}

The following datasets stem from samples collected at a pond in Victor
Ashe Park, 4901 Bradshaw Road, Knoxville, TN 37912. Analyses were
performed using each enzyme assay protocol. The standard curve data file
contains data for both protocols. The raw activity data files are split
by protocol. In these example datasets, all column names but `site.name'
and `std.type' are required for \emph{ezmmek} to function properly.
While the exact names of `site.name' and `std.type' are not required,
the user must input at least one descriptor column that is present in
both the standard curve dataset and the raw activity dataset for any
functions that rely on both datasets.

\hypertarget{standard-curve-data}{%
\paragraph{3.2.1 Standard curve data}\label{standard-curve-data}}

\begin{Shaded}
\begin{Highlighting}[]
\NormalTok{std\_data <{-}}\StringTok{ }\KeywordTok{read.csv}\NormalTok{(}\StringTok{"data/victor\_ashe\_std\_03172020.csv"}\NormalTok{)}
\KeywordTok{glimpse}\NormalTok{(std\_data)}
\CommentTok{\#> Observations: 10}
\CommentTok{\#> Variables: 5}
\CommentTok{\#> $ site.name     <fct> victor\_ashe, victor\_ashe, victor\_ashe, victor\_ashe, victor\_ashe, victor\_…}
\CommentTok{\#> $ std.type      <fct> amc, amc, amc, amc, amc, amc, amc, amc, amc, amc}
\CommentTok{\#> $ std.conc      <int> 0, 0, 1, 1, 2, 2, 3, 3, 4, 4}
\CommentTok{\#> $ homo.signal   <dbl> 4996.60, 4859.96, 229221.03, 222864.45, 429533.03, 378573.68, 592901.50,…}
\CommentTok{\#> $ buffer.signal <dbl> 413.67, 413.57, 401310.03, 387823.40, 794022.87, 822390.43, 1254915.12, …}
\end{Highlighting}
\end{Shaded}

\begin{Shaded}
\begin{Highlighting}[]
\NormalTok{site.name}\OperatorTok{:}\StringTok{ }\NormalTok{Name of sampling site, Victor Ashe Park, Knoxville, TN}
\NormalTok{std.type}\OperatorTok{:}\StringTok{ }\NormalTok{Type of standard, }\DecValTok{7}\OperatorTok{{-}}\NormalTok{Amino}\DecValTok{{-}4}\OperatorTok{{-}}\KeywordTok{methylcoumarin}\NormalTok{ (AMC)}
\NormalTok{std.conc}\OperatorTok{:}\StringTok{ }\NormalTok{Standard concentration, micromolar}
\NormalTok{homo.signal}\OperatorTok{:}\StringTok{ }\NormalTok{Signal from standard }\ControlFlowTok{in}\NormalTok{ homogenate, fluorescence units}
\NormalTok{buffer.signal}\OperatorTok{:}\StringTok{ }\NormalTok{Signal from standard }\ControlFlowTok{in}\NormalTok{ buffer, fluorescence units}
\end{Highlighting}
\end{Shaded}

\hypertarget{raw-activity-data-german}{%
\paragraph{3.2.2 Raw activity data,
``german''}\label{raw-activity-data-german}}

\begin{Shaded}
\begin{Highlighting}[]
\NormalTok{raw\_activity\_data\_german <{-}}\StringTok{ }\KeywordTok{read.csv}\NormalTok{(}\StringTok{"data/victor\_ashe\_sat\_german\_03172020.csv"}\NormalTok{)}
\KeywordTok{glimpse}\NormalTok{(raw\_activity\_data\_german)}
\CommentTok{\#> Observations: 13}
\CommentTok{\#> Variables: 13}
\CommentTok{\#> $ site.name    <fct> victor\_ashe, victor\_ashe, victor\_ashe, victor\_ashe, victor\_ashe, victor\_a…}
\CommentTok{\#> $ std.type     <fct> amc, amc, amc, amc, amc, amc, amc, amc, amc, amc, amc, amc, amc}
\CommentTok{\#> $ sub.type     <fct> l{-}leucine{-}amc, l{-}leucine{-}amc, l{-}leucine{-}amc, l{-}leucine{-}amc, l{-}leucine{-}amc…}
\CommentTok{\#> $ time         <int> 240, 240, 240, 240, 240, 240, 240, 240, 240, 240, 240, 240, 240}
\CommentTok{\#> $ sub.conc     <int> 0, 0, 0, 50, 50, 50, 100, 100, 100, 200, 200, 200, 400}
\CommentTok{\#> $ signal       <dbl> 4851.65, 5163.49, 5033.46, 295234.09, 162860.89, 159400.34, 267712.46, 28…}
\CommentTok{\#> $ replicate    <int> 1, 2, 3, 1, 2, 3, 1, 2, 3, 1, 2, 3, 1}
\CommentTok{\#> $ buffer.vol   <int> 1, 1, 1, 1, 1, 1, 1, 1, 1, 1, 1, 1, 1}
\CommentTok{\#> $ homo.vol     <int> 1, 1, 1, 1, 1, 1, 1, 1, 1, 1, 1, 1, 1}
\CommentTok{\#> $ soil.mass    <int> 1, 1, 1, 1, 1, 1, 1, 1, 1, 1, 1, 1, 1}
\CommentTok{\#> $ assay.vol    <int> 1, 1, 1, 1, 1, 1, 1, 1, 1, 1, 1, 1, 1}
\CommentTok{\#> $ homo.control <dbl> 4982.86, 4977.73, 4981.99, 4982.86, 4977.73, 4981.99, 4982.86, 4977.73, 4…}
\CommentTok{\#> $ sub.control  <dbl> 2137.63, 439.11, 846.38, 25705.65, 25577.37, 27237.21, 46103.79, 45554.96…}
\end{Highlighting}
\end{Shaded}

\begin{Shaded}
\begin{Highlighting}[]
\NormalTok{site.name}\OperatorTok{:}\StringTok{ }\NormalTok{Name of sampling site, Victor Ashe Park, Knoxville, TN}
\NormalTok{std.type}\OperatorTok{:}\StringTok{ }\NormalTok{Type of standard}
\NormalTok{time}\OperatorTok{:}\StringTok{ }\NormalTok{Timepoints at which signal was collected, minutes}
\NormalTok{signal}\OperatorTok{:}\StringTok{ }\NormalTok{Signal from assay, fluorescence units}
\NormalTok{sub.conc}\OperatorTok{:}\StringTok{ }\NormalTok{Substrate concentration, micromolar}
\NormalTok{replicate}\OperatorTok{:}\StringTok{ }\NormalTok{Sample replicated labeled numerically}
\NormalTok{homo.vol}\OperatorTok{:}\StringTok{ }\NormalTok{Homogenate volume, left as }\StringTok{\textquotesingle{}1\textquotesingle{}} \ControlFlowTok{for}\NormalTok{ water samples}
\NormalTok{soil.mass}\OperatorTok{:}\StringTok{ }\NormalTok{Soil mass, left as }\StringTok{\textquotesingle{}1\textquotesingle{}} \ControlFlowTok{for}\NormalTok{ water samples}
\NormalTok{assay.vol}\OperatorTok{:}\StringTok{ }\NormalTok{Assay volume, milliliters}
\NormalTok{homo.control}\OperatorTok{:}\StringTok{ }\NormalTok{Homogenate control, fluorescence units}
\NormalTok{sub.control}\OperatorTok{:}\StringTok{ }\NormalTok{Substrate control, flluorescence units}
\end{Highlighting}
\end{Shaded}

\hypertarget{raw-activity-data-steen}{%
\paragraph{3.2.3 Raw activity data,
``steen''}\label{raw-activity-data-steen}}

\begin{Shaded}
\begin{Highlighting}[]
\NormalTok{raw\_activity\_data\_steen <{-}}\StringTok{ }\KeywordTok{read.csv}\NormalTok{(}\StringTok{"data/victor\_ashe\_sat\_steen\_03172020.csv"}\NormalTok{)}
\KeywordTok{glimpse}\NormalTok{(raw\_activity\_data\_steen)}
\CommentTok{\#> Observations: 78}
\CommentTok{\#> Variables: 8}
\CommentTok{\#> $ site.name    <fct> victor\_ashe, victor\_ashe, victor\_ashe, victor\_ashe, victor\_ashe, victor\_a…}
\CommentTok{\#> $ std.type     <fct> amc, amc, amc, amc, amc, amc, amc, amc, amc, amc, amc, amc, amc, amc, amc…}
\CommentTok{\#> $ sub.type     <fct> l{-}leucine{-}amc, l{-}leucine{-}amc, l{-}leucine{-}amc, l{-}leucine{-}amc, l{-}leucine{-}amc…}
\CommentTok{\#> $ time         <int> 0, 20, 40, 60, 120, 240, 0, 20, 40, 60, 120, 240, 0, 20, 40, 60, 120, 240…}
\CommentTok{\#> $ sub.conc     <int> 0, 0, 0, 0, 0, 0, 0, 0, 0, 0, 0, 0, 0, 0, 0, 0, 0, 0, 50, 50, 50, 50, 50,…}
\CommentTok{\#> $ replicate    <int> 1, 1, 1, 1, 1, 1, 2, 2, 2, 2, 2, 2, 3, 3, 3, 3, 3, 3, 1, 1, 1, 1, 1, 1, 2…}
\CommentTok{\#> $ signal       <dbl> 4905.25, 4830.95, 4933.30, 4849.59, 4918.46, 4851.65, 4998.53, 5132.79, 5…}
\CommentTok{\#> $ kill.control <dbl> 8229.68, 8295.77, 8295.47, 8294.22, 8306.66, 8066.26, 8238.34, 8469.39, 8…}
\end{Highlighting}
\end{Shaded}

\begin{Shaded}
\begin{Highlighting}[]
\NormalTok{site.name}\OperatorTok{:}\StringTok{ }\NormalTok{Name of sampling site, Victor Ashe Park, Knoxville, TN}
\NormalTok{std.type}\OperatorTok{:}\StringTok{ }\NormalTok{Type of standard}
\NormalTok{time}\OperatorTok{:}\StringTok{ }\NormalTok{Timepoints at which signal was collected, minutes}
\NormalTok{signal}\OperatorTok{:}\StringTok{ }\NormalTok{Signal from assay, fluorescence units}
\NormalTok{sub.conc}\OperatorTok{:}\StringTok{ }\NormalTok{Substrate concentration, micromolar}
\NormalTok{replicate}\OperatorTok{:}\StringTok{ }\NormalTok{Sample replicated labeled numerically}
\NormalTok{kill.control}\OperatorTok{:}\StringTok{ }\NormalTok{Signal from autoclaved sample, fluorescence units}
\end{Highlighting}
\end{Shaded}

\hypertarget{visible-functions}{%
\subsubsection{3.3 Visible functions}\label{visible-functions}}

\emph{ezmmek} contains several functions that create new data.frame
objects. The functions build upon each other. The user can choose how
much analyses to perform, from generating standard curve models to
calculating km and vmax values. In descending order of parent functions
to child functions:

\begin{Shaded}
\begin{Highlighting}[]
\NormalTok{new\_ezmmek\_sat\_fit}
\NormalTok{new\_ezmmek\_act\_calibrate}
\NormalTok{new\_ezmmek\_act\_group}
\NormalTok{new\_ezmmek\_std\_group}
\end{Highlighting}
\end{Shaded}

\hypertarget{create-data.frame-object-of-class-new_ezmmek_sat_fit}{%
\paragraph{\texorpdfstring{3.3.1 Create data.frame object of class
\emph{new\_ezmmek\_sat\_fit}}{3.3.1 Create data.frame object of class new\_ezmmek\_sat\_fit}}\label{create-data.frame-object-of-class-new_ezmmek_sat_fit}}

\begin{Shaded}
\begin{Highlighting}[]
\NormalTok{new\_obj <{-}}\StringTok{ }\KeywordTok{new\_ezmmek\_sat\_fit}\NormalTok{(std.data.fn,}
\NormalTok{                              act.data.fn,}
\NormalTok{                              ...,}
                              \DataTypeTok{km =} \OtherTok{NULL}\NormalTok{,}
                              \DataTypeTok{vmax =} \OtherTok{NULL}\NormalTok{,}
                              \DataTypeTok{method =} \OtherTok{NA}\NormalTok{)}
\end{Highlighting}
\end{Shaded}

\begin{Shaded}
\begin{Highlighting}[]
\NormalTok{std.data.fn}\OperatorTok{:}\StringTok{ }\NormalTok{Standard curve data file as character string}
\NormalTok{act.data.fn}\OperatorTok{:}\StringTok{ }\NormalTok{Raw activity data file as character string}
\NormalTok{...}\OperatorTok{:}\StringTok{ }\NormalTok{User defined column names to join and group std.data.fn and act.data.fn}
\NormalTok{km}\OperatorTok{:}\StringTok{ }\NormalTok{Starting value to estimate km. Default value is median of }\StringTok{\textquotesingle{}sub.conc\textquotesingle{}}\NormalTok{ values}
\NormalTok{vmax}\OperatorTok{:}\StringTok{ }\NormalTok{Starting value to estimate vmax. Default value is max calibrated activity}
\NormalTok{method}\OperatorTok{:}\StringTok{ }\NormalTok{Enzyme assay protocol. Must define method as }\StringTok{\textquotesingle{}"steen"\textquotesingle{}}\NormalTok{ or }\StringTok{\textquotesingle{}"german"\textquotesingle{}}
\end{Highlighting}
\end{Shaded}

`\,``std.data.fn''\,' is the name of the standard curve data file.
`\,``act.data.fn''\,' is the name of the raw activity data file.
`\ldots{}' are the user-defined column names by which the standard curve
and raw activity data files are grouped and joined (\emph{i.e.}
descriptor columns like site name and type of fluorophore) to create a
new data.frame. The default numeric starting values of `km' and `vmax'
are those predicted by \emph{ezmmek}. `km' is calculated as the median
substrate concentration used during the experiment. `vmax' is calculated
as the maximum calibrated activity value. The user can assign their own
`km' and `vmax' starting values if they so choose. `method' must be set
equal to `\,``german''\,' or `\,``steen''\,'. The resulting dataframe
contains the descriptor columns, standard curve data, raw activity data,
calibrated activity data, and Michaelis-Menten fits.

\hypertarget{create-data.frame-object-of-class-new_ezmmek_calibrate}{%
\paragraph{\texorpdfstring{3.3.2 Create data.frame object of class
\emph{new\_ezmmek\_calibrate}}{3.3.2 Create data.frame object of class new\_ezmmek\_calibrate}}\label{create-data.frame-object-of-class-new_ezmmek_calibrate}}

\begin{Shaded}
\begin{Highlighting}[]
\NormalTok{new\_obj <{-}}\StringTok{ }\KeywordTok{new\_ezmmek\_act\_calibrate}\NormalTok{(std.data.fn,}
\NormalTok{                                    act.data.fn,}
\NormalTok{                                    ...,}
                                    \DataTypeTok{method =} \OtherTok{NA}\NormalTok{,}
                                    \DataTypeTok{columns =} \OtherTok{NULL}\NormalTok{)}
\end{Highlighting}
\end{Shaded}

\begin{Shaded}
\begin{Highlighting}[]
\NormalTok{std.data.fn}\OperatorTok{:}\StringTok{ }\NormalTok{Standard curve data file as character string}
\NormalTok{act.data.fn}\OperatorTok{:}\StringTok{ }\NormalTok{Raw activity data file as character string}
\NormalTok{...}\OperatorTok{:}\StringTok{ }\NormalTok{User defined column names to join and group std.data.fn and act.data.fn}
\NormalTok{method}\OperatorTok{:}\StringTok{ }\NormalTok{Enzyme assay protocol. Must define method as }\StringTok{\textquotesingle{}"steen"\textquotesingle{}}\NormalTok{ or }\StringTok{\textquotesingle{}"german"\textquotesingle{}}
\NormalTok{columns}\OperatorTok{:}\StringTok{ }\NormalTok{User defined column names to join and group std.data.fn and act.data.fn}
\end{Highlighting}
\end{Shaded}

`columns' carries over any `\ldots{}' arguments from parent functions.
If the user does not run this function within a parent function, the
`columns' argument can be ignored and left on the `NULL' default.
`\ldots{}' arguments must named if this function is used on its own. The
resulting data.frame contains the calibrated activities, but not the
Michaelis-Menten fits.

\hypertarget{create-data.frame-object-of-class-new_ezmmek_act_group}{%
\paragraph{\texorpdfstring{3.3.3 Create data.frame object of class
\emph{new\_ezmmek\_act\_group}}{3.3.3 Create data.frame object of class new\_ezmmek\_act\_group}}\label{create-data.frame-object-of-class-new_ezmmek_act_group}}

\begin{Shaded}
\begin{Highlighting}[]
\NormalTok{new\_obj <{-}}\StringTok{ }\KeywordTok{new\_ezmmek\_act\_group}\NormalTok{(act.data.fn,}
\NormalTok{                                ...,}
                                \DataTypeTok{method =} \OtherTok{NA}\NormalTok{,}
                                \DataTypeTok{columns =} \OtherTok{NULL}\NormalTok{)}
\end{Highlighting}
\end{Shaded}

\begin{Shaded}
\begin{Highlighting}[]
\NormalTok{act.data.fn}\OperatorTok{:}\StringTok{ }\NormalTok{Raw activity data file as character string}
\NormalTok{...}\OperatorTok{:}\StringTok{ }\NormalTok{User defined column names to join and group std.data.fn and act.data.fn}
\NormalTok{method}\OperatorTok{:}\StringTok{ }\NormalTok{Enzyme assay protocol. Must define method as }\StringTok{\textquotesingle{}"steen"\textquotesingle{}}\NormalTok{ or }\StringTok{\textquotesingle{}"german"\textquotesingle{}}
\NormalTok{columns}\OperatorTok{:}\StringTok{ }\NormalTok{User defined column names to join and group std.data.fn and act.data.fn}
\end{Highlighting}
\end{Shaded}

`columns' carries over any `\ldots{}' arguments from parent functions.
If the user does not run this function within a parent function, the
`columns' argument can be ignored and left on the `NULL' default.
`\ldots{}' arguments must be named if this function is used on its own.
The resulting data.frame contains grouped data of `act.data.fn', as
specified by column names.

\hypertarget{create-data.frame-object-of-class-new_ezmmek_std_group}{%
\paragraph{\texorpdfstring{3.3.4 Create data.frame object of class
\emph{new\_ezmmek\_std\_group}}{3.3.4 Create data.frame object of class new\_ezmmek\_std\_group}}\label{create-data.frame-object-of-class-new_ezmmek_std_group}}

\begin{Shaded}
\begin{Highlighting}[]
\NormalTok{new\_obj <{-}}\StringTok{ }\KeywordTok{new\_ezmmek\_std\_group}\NormalTok{(std.data.fn,}
\NormalTok{                                ...,}
                                \DataTypeTok{method =} \OtherTok{NA}\NormalTok{,}
                                \DataTypeTok{columns =} \OtherTok{NULL}\NormalTok{)}
\end{Highlighting}
\end{Shaded}

\begin{Shaded}
\begin{Highlighting}[]
\NormalTok{std.data.fn}\OperatorTok{:}\StringTok{ }\NormalTok{Standard curve data file as character string}
\NormalTok{...}\OperatorTok{:}\StringTok{ }\NormalTok{User defined column names to join and group std.data.fn and act.data.fn}
\NormalTok{method}\OperatorTok{:}\StringTok{ }\NormalTok{Enzyme assay protocol. Must define method as }\StringTok{\textquotesingle{}"steen"\textquotesingle{}}\NormalTok{ or }\StringTok{\textquotesingle{}"german"\textquotesingle{}}
\NormalTok{columns}\OperatorTok{:}\StringTok{ }\NormalTok{User defined column names to group std.data.fn}
\end{Highlighting}
\end{Shaded}

`columns' carries over any `\ldots{}' arguments from parent functions.
If the user does not run this function within a parent function, the
`columns' argument can be ignored and left on the `NULL' default.
`\ldots{}' arguments must be named if this function is used on its own.
The resulting data.frame contains grouped `std.data.fn' data, as
specified by column names, and standard curve linear models for each
group.

\hypertarget{example-analyses}{%
\subsection{4 Example analyses}\label{example-analyses}}

\hypertarget{new_ezmmek_sat_fit}{%
\subsubsection{\texorpdfstring{4.1
\emph{new\_ezmmek\_sat\_fit}}{4.1 new\_ezmmek\_sat\_fit}}\label{new_ezmmek_sat_fit}}

\hypertarget{new_ezmmek_sat_fit-german}{%
\paragraph{\texorpdfstring{4.1.1 \emph{new\_ezmmek\_sat\_fit},
``german''}{4.1.1 new\_ezmmek\_sat\_fit, ``german''}}\label{new_ezmmek_sat_fit-german}}

\begin{Shaded}
\begin{Highlighting}[]
\NormalTok{new\_ezmmek\_sat\_fit\_obj <{-}}\StringTok{ }\KeywordTok{new\_ezmmek\_sat\_fit}\NormalTok{(}\StringTok{"data/victor\_ashe\_std\_03172020.csv"}\NormalTok{, }
                                     \StringTok{"data/victor\_ashe\_sat\_german\_03172020.csv"}\NormalTok{, }
\NormalTok{                                     site.name, }
\NormalTok{                                     std.type,}
                                     \DataTypeTok{km =} \OtherTok{NULL}\NormalTok{,}
                                     \DataTypeTok{vmax =} \OtherTok{NULL}\NormalTok{,}
                                     \DataTypeTok{method =} \StringTok{"german"}\NormalTok{)}
\CommentTok{\#> Joining, by = c("site.name", "std.type")}

\KeywordTok{glimpse}\NormalTok{(new\_ezmmek\_sat\_fit\_obj)}
\CommentTok{\#> Observations: 13}
\CommentTok{\#> Variables: 29}
\CommentTok{\#> $ site.name               <fct> victor\_ashe, victor\_ashe, victor\_ashe, victor\_ashe, victor\_ash…}
\CommentTok{\#> $ std.type                <fct> amc, amc, amc, amc, amc, amc, amc, amc, amc, amc, amc, amc, amc}
\CommentTok{\#> $ sub.type                <fct> l{-}leucine{-}amc, l{-}leucine{-}amc, l{-}leucine{-}amc, l{-}leucine{-}amc, l{-}…}
\CommentTok{\#> $ time                    <int> 240, 240, 240, 240, 240, 240, 240, 240, 240, 240, 240, 240, 240}
\CommentTok{\#> $ sub.conc                <int> 0, 0, 0, 50, 50, 50, 100, 100, 100, 200, 200, 200, 400}
\CommentTok{\#> $ signal                  <dbl> 4851.65, 5163.49, 5033.46, 295234.09, 162860.89, 159400.34, 26…}
\CommentTok{\#> $ replicate               <int> 1, 2, 3, 1, 2, 3, 1, 2, 3, 1, 2, 3, 1}
\CommentTok{\#> $ buffer.vol              <int> 1, 1, 1, 1, 1, 1, 1, 1, 1, 1, 1, 1, 1}
\CommentTok{\#> $ homo.vol                <int> 1, 1, 1, 1, 1, 1, 1, 1, 1, 1, 1, 1, 1}
\CommentTok{\#> $ soil.mass               <int> 1, 1, 1, 1, 1, 1, 1, 1, 1, 1, 1, 1, 1}
\CommentTok{\#> $ assay.vol               <int> 1, 1, 1, 1, 1, 1, 1, 1, 1, 1, 1, 1, 1}
\CommentTok{\#> $ homo.control            <dbl> 4982.86, 4977.73, 4981.99, 4982.86, 4977.73, 4981.99, 4982.86,…}
\CommentTok{\#> $ sub.control             <dbl> 2137.63, 439.11, 846.38, 25705.65, 25577.37, 27237.21, 46103.7…}
\CommentTok{\#> $ std.raw.data.g          <list> [<tbl\_df[10 x 3]>, <tbl\_df[10 x 3]>, <tbl\_df[10 x 3]>, <tbl\_d…}
\CommentTok{\#> $ std.lm.homo.obj         <list> [<{-}8853.539, 215506.959, 13850.14, 13713.50, 22567.61, 16211.…}
\CommentTok{\#> $ std.lm.homo.slope       <dbl> 215507, 215507, 215507, 215507, 215507, 215507, 215507, 215507…}
\CommentTok{\#> $ std.lm.homo.intercept   <dbl> {-}8853.539, {-}8853.539, {-}8853.539, {-}8853.539, {-}8853.539, {-}8853.5…}
\CommentTok{\#> $ std.lm.buffer.obj       <list> [<{-}20604.44, 428683.42, 21018.107, 21018.007, {-}6768.956, {-}202…}
\CommentTok{\#> $ std.lm.buffer.slope     <dbl> 428683.4, 428683.4, 428683.4, 428683.4, 428683.4, 428683.4, 42…}
\CommentTok{\#> $ std.lm.buffer.intercept <dbl> {-}20604.44, {-}20604.44, {-}20604.44, {-}20604.44, {-}20604.44, {-}20604.…}
\CommentTok{\#> $ quench.coef             <dbl> 0.5027182, 0.5027182, 0.5027182, 0.5027182, 0.5027182, 0.50271…}
\CommentTok{\#> $ emission.coef           <dbl> 215507, 215507, 215507, 215507, 215507, 215507, 215507, 215507…}
\CommentTok{\#> $ net.signal              <dbl> {-}2398.63109, {-}69.59881, {-}743.99660, 551658.02816, 288481.59877…}
\CommentTok{\#> $ activity                <dbl> {-}4.637575e{-}05, {-}1.345641e{-}06, {-}1.438462e{-}05, 1.066590e{-}02, 5.5…}
\CommentTok{\#> $ activity.m              <dbl> {-}0.000020702, {-}0.000020702, {-}0.000020702, 0.007218568, 0.00721…}
\CommentTok{\#> $ activity.sd             <dbl> 2.317023e{-}05, 2.317023e{-}05, 2.317023e{-}05, 2.986619e{-}03, 2.9866…}
\CommentTok{\#> $ mm.fit.obj              <list> [<function () , resid, function () , rhs, function () , form,…}
\CommentTok{\#> $ km                      <dbl> 20.52471, 20.52471, 20.52471, 20.52471, 20.52471, 20.52471, 20…}
\CommentTok{\#> $ vmax                    <dbl> 0.0107279, 0.0107279, 0.0107279, 0.0107279, 0.0107279, 0.01072…}
\end{Highlighting}
\end{Shaded}

\hypertarget{new_ezmmek_sat_fit-steen}{%
\paragraph{\texorpdfstring{4.1.2 \emph{new\_ezmmek\_sat\_fit},
``steen''}{4.1.2 new\_ezmmek\_sat\_fit, ``steen''}}\label{new_ezmmek_sat_fit-steen}}

\begin{Shaded}
\begin{Highlighting}[]
\NormalTok{new\_ezmmek\_sat\_fit\_obj <{-}}\StringTok{ }\KeywordTok{new\_ezmmek\_sat\_fit}\NormalTok{(}\StringTok{"data/victor\_ashe\_std\_03172020.csv"}\NormalTok{,}
                                             \StringTok{"data/victor\_ashe\_sat\_steen\_03172020.csv"}\NormalTok{, }
\NormalTok{                                             site.name, }
\NormalTok{                                             std.type,}
                                             \DataTypeTok{km =} \OtherTok{NULL}\NormalTok{,}
                                             \DataTypeTok{vmax =} \OtherTok{NULL}\NormalTok{,}
                                             \DataTypeTok{method =} \StringTok{"steen"}\NormalTok{)}
\CommentTok{\#> Joining, by = c("site.name", "std.type")}

\KeywordTok{glimpse}\NormalTok{(new\_ezmmek\_sat\_fit\_obj)}
\CommentTok{\#> Observations: 13}
\CommentTok{\#> Variables: 16}
\CommentTok{\#> $ site.name             <fct> victor\_ashe, victor\_ashe, victor\_ashe, victor\_ashe, victor\_ashe,…}
\CommentTok{\#> $ std.type              <fct> amc, amc, amc, amc, amc, amc, amc, amc, amc, amc, amc, amc, amc}
\CommentTok{\#> $ sub.type              <fct> l{-}leucine{-}amc, l{-}leucine{-}amc, l{-}leucine{-}amc, l{-}leucine{-}amc, l{-}le…}
\CommentTok{\#> $ sub.conc              <int> 0, 0, 0, 50, 50, 50, 100, 100, 100, 200, 200, 200, 400}
\CommentTok{\#> $ replicate             <int> 1, 2, 3, 1, 2, 3, 1, 2, 3, 1, 2, 3, 1}
\CommentTok{\#> $ std.raw.data.s        <list> [<tbl\_df[10 x 3]>, <tbl\_df[10 x 3]>, <tbl\_df[10 x 3]>, <tbl\_df[…}
\CommentTok{\#> $ std.lm.homo.obj       <list> [<{-}8853.539, 215506.959, 13850.14, 13713.50, 22567.61, 16211.03…}
\CommentTok{\#> $ std.lm.homo.slope     <dbl> 215507, 215507, 215507, 215507, 215507, 215507, 215507, 215507, …}
\CommentTok{\#> $ std.lm.homo.intercept <dbl> {-}8853.539, {-}8853.539, {-}8853.539, {-}8853.539, {-}8853.539, {-}8853.539…}
\CommentTok{\#> $ act.calibrated.data   <list<df[,4]>> 0.000000e+00, 2.000000e+01, 4.000000e+01, 6.000000e+01,…}
\CommentTok{\#> $ activity              <dbl> 3.238377e{-}06, {-}3.605641e{-}07, {-}6.439538e{-}06, 4.804954e{-}03, 2.7291…}
\CommentTok{\#> $ activity.m            <dbl> {-}1.187242e{-}06, {-}1.187242e{-}06, {-}1.187242e{-}06, 3.188943e{-}03, 3.188…}
\CommentTok{\#> $ activity.sd           <dbl> 4.891632e{-}06, 4.891632e{-}06, 4.891632e{-}06, 1.442172e{-}03, 1.442172…}
\CommentTok{\#> $ mm.fit.obj            <list> [<function () , resid, function () , rhs, function () , form, f…}
\CommentTok{\#> $ km                    <dbl> 26.34536, 26.34536, 26.34536, 26.34536, 26.34536, 26.34536, 26.3…}
\CommentTok{\#> $ vmax                  <dbl> 0.004836086, 0.004836086, 0.004836086, 0.004836086, 0.004836086,…}
\end{Highlighting}
\end{Shaded}

\hypertarget{new_ezmmek_act_calibrate}{%
\subsubsection{\texorpdfstring{4.2
\emph{new\_ezmmek\_act\_calibrate}}{4.2 new\_ezmmek\_act\_calibrate}}\label{new_ezmmek_act_calibrate}}

\hypertarget{new_ezmmek_act_calibrate-german}{%
\paragraph{\texorpdfstring{4.2.1 \emph{new\_ezmmek\_act\_calibrate},
``german''}{4.2.1 new\_ezmmek\_act\_calibrate, ``german''}}\label{new_ezmmek_act_calibrate-german}}

\begin{Shaded}
\begin{Highlighting}[]
\NormalTok{new\_ezmmek\_act\_calibrate\_obj <{-}}\StringTok{ }\KeywordTok{new\_ezmmek\_act\_calibrate}\NormalTok{(}\StringTok{"data/victor\_ashe\_std\_03172020.csv"}\NormalTok{,}
                                                 \StringTok{"data/victor\_ashe\_sat\_german\_03172020.csv"}\NormalTok{,}
\NormalTok{                                                 site.name,}
\NormalTok{                                                 std.type,}
                                                 \DataTypeTok{method =} \StringTok{"german"}\NormalTok{,}
                                                 \DataTypeTok{columns =} \OtherTok{NULL}\NormalTok{)}
\CommentTok{\#> Joining, by = c("site.name", "std.type")}

\KeywordTok{glimpse}\NormalTok{(new\_ezmmek\_act\_calibrate\_obj)}
\CommentTok{\#> Observations: 13}
\CommentTok{\#> Variables: 26}
\CommentTok{\#> $ site.name               <fct> victor\_ashe, victor\_ashe, victor\_ashe, victor\_ashe, victor\_ash…}
\CommentTok{\#> $ std.type                <fct> amc, amc, amc, amc, amc, amc, amc, amc, amc, amc, amc, amc, amc}
\CommentTok{\#> $ sub.type                <fct> l{-}leucine{-}amc, l{-}leucine{-}amc, l{-}leucine{-}amc, l{-}leucine{-}amc, l{-}…}
\CommentTok{\#> $ time                    <int> 240, 240, 240, 240, 240, 240, 240, 240, 240, 240, 240, 240, 240}
\CommentTok{\#> $ sub.conc                <int> 0, 0, 0, 50, 50, 50, 100, 100, 100, 200, 200, 200, 400}
\CommentTok{\#> $ signal                  <dbl> 4851.65, 5163.49, 5033.46, 295234.09, 162860.89, 159400.34, 26…}
\CommentTok{\#> $ replicate               <int> 1, 2, 3, 1, 2, 3, 1, 2, 3, 1, 2, 3, 1}
\CommentTok{\#> $ buffer.vol              <int> 1, 1, 1, 1, 1, 1, 1, 1, 1, 1, 1, 1, 1}
\CommentTok{\#> $ homo.vol                <int> 1, 1, 1, 1, 1, 1, 1, 1, 1, 1, 1, 1, 1}
\CommentTok{\#> $ soil.mass               <int> 1, 1, 1, 1, 1, 1, 1, 1, 1, 1, 1, 1, 1}
\CommentTok{\#> $ assay.vol               <int> 1, 1, 1, 1, 1, 1, 1, 1, 1, 1, 1, 1, 1}
\CommentTok{\#> $ homo.control            <dbl> 4982.86, 4977.73, 4981.99, 4982.86, 4977.73, 4981.99, 4982.86,…}
\CommentTok{\#> $ sub.control             <dbl> 2137.63, 439.11, 846.38, 25705.65, 25577.37, 27237.21, 46103.7…}
\CommentTok{\#> $ std.raw.data.g          <list> [<tbl\_df[10 x 3]>, <tbl\_df[10 x 3]>, <tbl\_df[10 x 3]>, <tbl\_d…}
\CommentTok{\#> $ std.lm.homo.obj         <list> [<{-}8853.539, 215506.959, 13850.14, 13713.50, 22567.61, 16211.…}
\CommentTok{\#> $ std.lm.homo.slope       <dbl> 215507, 215507, 215507, 215507, 215507, 215507, 215507, 215507…}
\CommentTok{\#> $ std.lm.homo.intercept   <dbl> {-}8853.539, {-}8853.539, {-}8853.539, {-}8853.539, {-}8853.539, {-}8853.5…}
\CommentTok{\#> $ std.lm.buffer.obj       <list> [<{-}20604.44, 428683.42, 21018.107, 21018.007, {-}6768.956, {-}202…}
\CommentTok{\#> $ std.lm.buffer.slope     <dbl> 428683.4, 428683.4, 428683.4, 428683.4, 428683.4, 428683.4, 42…}
\CommentTok{\#> $ std.lm.buffer.intercept <dbl> {-}20604.44, {-}20604.44, {-}20604.44, {-}20604.44, {-}20604.44, {-}20604.…}
\CommentTok{\#> $ quench.coef             <dbl> 0.5027182, 0.5027182, 0.5027182, 0.5027182, 0.5027182, 0.50271…}
\CommentTok{\#> $ emission.coef           <dbl> 215507, 215507, 215507, 215507, 215507, 215507, 215507, 215507…}
\CommentTok{\#> $ net.signal              <dbl> {-}2398.63109, {-}69.59881, {-}743.99660, 551658.02816, 288481.59877…}
\CommentTok{\#> $ activity                <dbl> {-}4.637575e{-}05, {-}1.345641e{-}06, {-}1.438462e{-}05, 1.066590e{-}02, 5.5…}
\CommentTok{\#> $ activity.m              <dbl> {-}0.000020702, {-}0.000020702, {-}0.000020702, 0.007218568, 0.00721…}
\CommentTok{\#> $ activity.sd             <dbl> 2.317023e{-}05, 2.317023e{-}05, 2.317023e{-}05, 2.986619e{-}03, 2.9866…}
\end{Highlighting}
\end{Shaded}

\hypertarget{new_ezmmek_act_calibrate-steen-method}{%
\paragraph{\texorpdfstring{4.2.2 \emph{new\_ezmmek\_act\_calibrate},
``steen''
method}{4.2.2 new\_ezmmek\_act\_calibrate, ``steen'' method}}\label{new_ezmmek_act_calibrate-steen-method}}

\begin{Shaded}
\begin{Highlighting}[]
\NormalTok{new\_ezmmek\_act\_calibrate\_obj <{-}}\StringTok{ }\KeywordTok{new\_ezmmek\_act\_calibrate}\NormalTok{(}\StringTok{"data/victor\_ashe\_std\_03172020.csv"}\NormalTok{,}
                                                 \StringTok{"data/victor\_ashe\_sat\_steen\_03172020.csv"}\NormalTok{,}
\NormalTok{                                                 site.name,}
\NormalTok{                                                 std.type,}
                                                 \DataTypeTok{method =} \StringTok{"steen"}\NormalTok{,}
                                                 \DataTypeTok{columns =} \OtherTok{NULL}\NormalTok{)}
\CommentTok{\#> Joining, by = c("site.name", "std.type")}

\KeywordTok{glimpse}\NormalTok{(new\_ezmmek\_act\_calibrate\_obj)}
\CommentTok{\#> Observations: 13}
\CommentTok{\#> Variables: 13}
\CommentTok{\#> $ site.name             <fct> victor\_ashe, victor\_ashe, victor\_ashe, victor\_ashe, victor\_ashe,…}
\CommentTok{\#> $ std.type              <fct> amc, amc, amc, amc, amc, amc, amc, amc, amc, amc, amc, amc, amc}
\CommentTok{\#> $ sub.type              <fct> l{-}leucine{-}amc, l{-}leucine{-}amc, l{-}leucine{-}amc, l{-}leucine{-}amc, l{-}le…}
\CommentTok{\#> $ sub.conc              <int> 0, 0, 0, 50, 50, 50, 100, 100, 100, 200, 200, 200, 400}
\CommentTok{\#> $ replicate             <int> 1, 2, 3, 1, 2, 3, 1, 2, 3, 1, 2, 3, 1}
\CommentTok{\#> $ std.raw.data.s        <list> [<tbl\_df[10 x 3]>, <tbl\_df[10 x 3]>, <tbl\_df[10 x 3]>, <tbl\_df[…}
\CommentTok{\#> $ std.lm.homo.obj       <list> [<{-}8853.539, 215506.959, 13850.14, 13713.50, 22567.61, 16211.03…}
\CommentTok{\#> $ std.lm.homo.slope     <dbl> 215507, 215507, 215507, 215507, 215507, 215507, 215507, 215507, …}
\CommentTok{\#> $ std.lm.homo.intercept <dbl> {-}8853.539, {-}8853.539, {-}8853.539, {-}8853.539, {-}8853.539, {-}8853.539…}
\CommentTok{\#> $ act.calibrated.data   <list<df[,4]>> 0.000000e+00, 2.000000e+01, 4.000000e+01, 6.000000e+01,…}
\CommentTok{\#> $ activity              <dbl> 3.238377e{-}06, {-}3.605641e{-}07, {-}6.439538e{-}06, 4.804954e{-}03, 2.7291…}
\CommentTok{\#> $ activity.m            <dbl> {-}1.187242e{-}06, {-}1.187242e{-}06, {-}1.187242e{-}06, 3.188943e{-}03, 3.188…}
\CommentTok{\#> $ activity.sd           <dbl> 4.891632e{-}06, 4.891632e{-}06, 4.891632e{-}06, 1.442172e{-}03, 1.442172…}
\end{Highlighting}
\end{Shaded}

\hypertarget{new_ezmmek_act_group}{%
\subsubsection{\texorpdfstring{4.3
\emph{new\_ezmmek\_act\_group}}{4.3 new\_ezmmek\_act\_group}}\label{new_ezmmek_act_group}}

\hypertarget{new_ezmmek_act_group-german}{%
\paragraph{\texorpdfstring{4.3.1 \emph{new\_ezmmek\_act\_group},
``german''}{4.3.1 new\_ezmmek\_act\_group, ``german''}}\label{new_ezmmek_act_group-german}}

\begin{Shaded}
\begin{Highlighting}[]
\NormalTok{new\_ezmmek\_act\_group\_obj <{-}}\StringTok{ }\KeywordTok{new\_ezmmek\_act\_group}\NormalTok{(}\StringTok{"data/victor\_ashe\_sat\_german\_03172020.csv"}\NormalTok{,}
\NormalTok{                                                 site.name,}
\NormalTok{                                                 std.type,}
                                                 \DataTypeTok{method =} \StringTok{"german"}\NormalTok{,}
                                                 \DataTypeTok{columns =} \OtherTok{NULL}\NormalTok{)}

\KeywordTok{glimpse}\NormalTok{(new\_ezmmek\_act\_group\_obj)}
\CommentTok{\#> Observations: 1}
\CommentTok{\#> Variables: 3}
\CommentTok{\#> $ site.name      <fct> victor\_ashe}
\CommentTok{\#> $ std.type       <fct> amc}
\CommentTok{\#> $ act.raw.data.g <list> [<tbl\_df[13 x 11]>]}
\end{Highlighting}
\end{Shaded}

\hypertarget{new_ezmmek_act_group-steen}{%
\paragraph{\texorpdfstring{4.3.2 \emph{new\_ezmmek\_act\_group},
``steen''}{4.3.2 new\_ezmmek\_act\_group, ``steen''}}\label{new_ezmmek_act_group-steen}}

\begin{Shaded}
\begin{Highlighting}[]
\NormalTok{new\_ezmmek\_act\_group\_obj <{-}}\StringTok{ }\KeywordTok{new\_ezmmek\_act\_group}\NormalTok{(}\StringTok{"data/victor\_ashe\_sat\_steen\_03172020.csv"}\NormalTok{,}
\NormalTok{                                                 site.name,}
\NormalTok{                                                 std.type,}
                                                 \DataTypeTok{method =} \StringTok{"steen"}\NormalTok{,}
                                                 \DataTypeTok{columns =} \OtherTok{NULL}\NormalTok{)}

\KeywordTok{glimpse}\NormalTok{(new\_ezmmek\_act\_group\_obj)}
\CommentTok{\#> Observations: 1}
\CommentTok{\#> Variables: 3}
\CommentTok{\#> $ site.name      <fct> victor\_ashe}
\CommentTok{\#> $ std.type       <fct> amc}
\CommentTok{\#> $ act.raw.data.s <list> [<tbl\_df[78 x 6]>]}
\end{Highlighting}
\end{Shaded}

\hypertarget{new_ezmmek_std_group}{%
\subsubsection{\texorpdfstring{4.4
\emph{new\_ezmmek\_std\_group}}{4.4 new\_ezmmek\_std\_group}}\label{new_ezmmek_std_group}}

\hypertarget{new_ezmmek_std_group-german}{%
\paragraph{\texorpdfstring{4.4.1 \emph{new\_ezmmek\_std\_group},
``german''}{4.4.1 new\_ezmmek\_std\_group, ``german''}}\label{new_ezmmek_std_group-german}}

\begin{Shaded}
\begin{Highlighting}[]
\NormalTok{new\_ezmmek\_std\_group\_obj <{-}}\StringTok{ }\KeywordTok{new\_ezmmek\_std\_group}\NormalTok{(}\StringTok{"data/victor\_ashe\_std\_03172020.csv"}\NormalTok{,}
\NormalTok{                                                 site.name,}
\NormalTok{                                                 std.type,}
                                                 \DataTypeTok{method =} \StringTok{"german"}\NormalTok{,}
                                                 \DataTypeTok{columns =} \OtherTok{NULL}\NormalTok{)}

\KeywordTok{glimpse}\NormalTok{(new\_ezmmek\_std\_group\_obj)}
\CommentTok{\#> Observations: 1}
\CommentTok{\#> Variables: 10}
\CommentTok{\#> $ site.name               <fct> victor\_ashe}
\CommentTok{\#> $ std.type                <fct> amc}
\CommentTok{\#> $ std.raw.data.g          <list> [<tbl\_df[10 x 3]>]}
\CommentTok{\#> $ std.lm.homo.obj         <list> [<{-}8853.539, 215506.959, 13850.14, 13713.50, 22567.61, 16211.…}
\CommentTok{\#> $ std.lm.homo.slope       <dbl> 215507}
\CommentTok{\#> $ std.lm.homo.intercept   <dbl> {-}8853.539}
\CommentTok{\#> $ std.lm.buffer.obj       <list> [<{-}20604.44, 428683.42, 21018.107, 21018.007, {-}6768.956, {-}202…}
\CommentTok{\#> $ std.lm.buffer.slope     <dbl> 428683.4}
\CommentTok{\#> $ std.lm.buffer.intercept <dbl> {-}20604.44}
\CommentTok{\#> $ quench.coef             <dbl> 0.5027182}
\end{Highlighting}
\end{Shaded}

\hypertarget{new_ezmmek_std_group-steen}{%
\paragraph{\texorpdfstring{4.4.2 \emph{new\_ezmmek\_std\_group},
``steen''}{4.4.2 new\_ezmmek\_std\_group, ``steen''}}\label{new_ezmmek_std_group-steen}}

\begin{Shaded}
\begin{Highlighting}[]
\NormalTok{new\_ezmmek\_std\_group\_obj <{-}}\StringTok{ }\KeywordTok{new\_ezmmek\_std\_group}\NormalTok{(}\StringTok{"data/victor\_ashe\_std\_03172020.csv"}\NormalTok{,}
\NormalTok{                                                 site.name,}
\NormalTok{                                                 std.type,}
                                                 \DataTypeTok{method =} \StringTok{"german"}\NormalTok{,}
                                                 \DataTypeTok{columns =} \OtherTok{NULL}\NormalTok{)}
\KeywordTok{glimpse}\NormalTok{(new\_ezmmek\_std\_group\_obj)}
\CommentTok{\#> Observations: 1}
\CommentTok{\#> Variables: 10}
\CommentTok{\#> $ site.name               <fct> victor\_ashe}
\CommentTok{\#> $ std.type                <fct> amc}
\CommentTok{\#> $ std.raw.data.g          <list> [<tbl\_df[10 x 3]>]}
\CommentTok{\#> $ std.lm.homo.obj         <list> [<{-}8853.539, 215506.959, 13850.14, 13713.50, 22567.61, 16211.…}
\CommentTok{\#> $ std.lm.homo.slope       <dbl> 215507}
\CommentTok{\#> $ std.lm.homo.intercept   <dbl> {-}8853.539}
\CommentTok{\#> $ std.lm.buffer.obj       <list> [<{-}20604.44, 428683.42, 21018.107, 21018.007, {-}6768.956, {-}202…}
\CommentTok{\#> $ std.lm.buffer.slope     <dbl> 428683.4}
\CommentTok{\#> $ std.lm.buffer.intercept <dbl> {-}20604.44}
\CommentTok{\#> $ quench.coef             <dbl> 0.5027182}
\end{Highlighting}
\end{Shaded}

\hypertarget{methods-for-new_ezmmek-objects}{%
\subsection{\texorpdfstring{5 Methods for \emph{new\_ezmmek}
objects}{5 Methods for new\_ezmmek objects}}\label{methods-for-new_ezmmek-objects}}

\hypertarget{plot}{%
\subsubsection{\texorpdfstring{5.1 \emph{plot}}{5.1 plot}}\label{plot}}

\hypertarget{summary}{%
\subsubsection{\texorpdfstring{5.2
\emph{summary}}{5.2 summary}}\label{summary}}

\hypertarget{coef}{%
\subsubsection{\texorpdfstring{5.3 \emph{coef}}{5.3 coef}}\label{coef}}

\hypertarget{references}{%
\subsection{6 References}\label{references}}

German \emph{et al.} (2011) \url{doi:10.1016/j.soilbio.2011.03.017}\\
Sinsabaugh \emph{et al.} (2014) \url{doi:10.1007/s10533-014-0030-y}\\
Steen and Arnosti (2011) \url{doi:10.1016/j.marchem.2010.10.006}

\hypertarget{authors}{%
\subsection{7 Authors}\label{authors}}

Christopher L. Cook
(\href{mailto:ccook62@vols.utk.edu}{\nolinkurl{ccook62@vols.utk.edu}})
and Andrew D. Steen

\hypertarget{license}{%
\subsection{8 License}\label{license}}

This project is licensed under AGPL-3.

\end{document}
